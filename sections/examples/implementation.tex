\section{Implementation}

In this section, we describe our prototype of \TheName. We implement \TheName
with more than \texttt{1.6k} lines C code
on an Armv8 juno board based on Linux. In our prototype of \TheName, we analyze the semantics of the 60 commonly used
\syscall{}s. For each \syscall{}, \TheName generates a description including  return address, process id and its effects. We also use two 16 MB buffers to temporarily store
the \syscall{} records and then dump this buffer to files when it is full. 

In the \textit{Core Hook} part, As Figure \ref{fig:core-hook-desgin} demonstrates, the entrance hook and the exit hook inspect the parameters and the impact of a syscall, respectively. At the entrance hook, \TheName stores the syscall number and the first two parameters (i.e., saved in \texttt{x0} and \texttt{x1}) into a hashmap. The key for the hashmap is the \texttt{thread\_id} because a thread can only issue a single syscall until exit. At the exit hook, \TheName identifies the type of current syscall and applies specific capturing rules.

To avoid the uncertain influence of using Linux APIs (e.g., different versions with different implementations), I re-write some interfaces relevant to string operations (e.g., \texttt{strncpy}, \texttt{sprintf}). Besides, we further apply a simple encoding rule (i.e., encode the numbers to bytes) to reduce the recording size. 

The \textit{Filter} gets the filter conditions passed by the user, \TheName supports
the filter of process id, process name, parent process id, and parent process name. The syscall satisfying all passed conditions then delivers to \textit{Record Buffer}.

The \textit{Record Buffer} maintains two global buffers to store records temporarily. Considering the parallelism of syscalls (i.e., different processor can tickle syscalls at the same time), writing to buffer or writing to file needs locks to prevent data races. I choose to use spinlock before each writes operation to buffer. For file writing, I use a pair of spinlock to protect a global flag and do not set any lock to protect writing to the file itself. Although simultaneous writing to file is unacceptable, \TheName usually can not fill a buffer before the last writing is finished. The evaluation about the writing file provides in \S\ref{subsec:eva-flowaccuracy} 


\TheName only incurs a low overhead in both time and space. Detailed measurement is presented
in \S \ref{subsec:eva-Efficiency}.

