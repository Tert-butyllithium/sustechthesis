\section{Intoduction}
The program often fails. To sufficiently understand and prevent failures,
developers requires firstly reproduce these bugs, which ensures the same output
and bugs. However, directly
re-exection is not suitable for non-deterministic failures, as they may not
appear in a re-execution procedure. Non-deterministic failures are the
consequence of non-deterministic instructions. 

Instructions for running a program can be divided into two categories. One is
deterministic, which means the behavior of deterministic instruction is determined in each
execution. The other type is non-deterministic, meaning that execution in
different situations will have different results. Although most of the CPU
execution is deterministic (e.g., \texttt{ADD}), non-deterministic instructions (e.g., get user input) are also pervasive.
Typical sources of nondeterminism include system calls, interrupts, signals, and
data races for concurrency programs \cite{ronsse_recplay_1999}. All these non-deterministic events can be futher classified into two types:
inconstancy of the data flow - for example, certain system calls such as
\texttt{getrandom} and \texttt{getpid}, and inconstancy of the control flow
- for example, concurrency bug due to memory access in inconsistent order \cite{getrandom2}.

Record-and-replay is a type of approaches that addresses this challenge. Most
Record-and-replay systems work by first recording non-deterministic events
during the original run of a program and then substituting these records during
subsequent re-execution. Record-and-replay system could ultimately guarante that
each replay will be identical with the initial version. The fact that a number
of replay systems have been built and put into use in recent years illustrates
the value of record-and-replay systems in practice \cite{203227,replay_survey,altekar_odr_2009,bhansali_framework_2006}.

% There are several ways to capture calls online at runtime: \textbf{PinPlay}, \textbf{REPT}, \textbf{rr}

There is a rich amount of research on record-and-replay systems, and we can find
their various treatments of non-deterministic records. Early record-and-replay
systems tend to use virtualization techniques so as to observe and record the
entire program non-deterministically on the hypervisior, but the virtual machine
is very heavy \cite{dunlap_revirt_2003, dunlap_smp-revirt_2008}. Some systems
use dynamic binary instrumentation to get the results after running each
instruction, but this is very inefficient \cite{bhansali_framework_2006}. There
are some other systems that choose not to record at runtime in order to address
the expensive cost of recording; instead, they infer these non-deterministic
events based on the control flow and other information collected
\cite{altekar_odr_2009,cui_rept_2018}. However, inference often does not
reproduce program execution as faithfully as records, and the time required for
inference, which in the worst case is a search of the entire space, is a problem
\cite{replay_survey}. There are also systems that use custom hardware, which
inevitably affects its usefulness in practice \cite{montesinos_capo_2009}.
Recently there have been some practical systems that have adopted tools provided
by Linux for tracing, thus achieving better efficiency. Nevertheless, it still
introduces a considerable overhead (50\%) and is therefore only available
when the developer exactly needs to record and replay \cite{203227}.

This thesis focuses on the data record part of record-and-replay systems, precisely, the recording of non-deterministic events caused by system calls. I argue that a \textit{practical} record system should  (1) run online, meaning that the recording has little performance impact on the execution of the target program, (2) log all data without any omission, (3) work on commercial off-the-shelf hardware, (4) not require any modification to the target program, and also (5) not require any modification to the kernel.

In this thesis, I propose \TheName, a practical solution for syscall capturing. 
It works with unmodified Linux programs on commercial off-the-shelf (OTS) hardware. My original design was on the ARM platform, but the system can be applied to other platforms as well (e.g. x86, riscv). I demonstrat its usefulness on both x86 and ARM platforms. \TheName consists of three component: \CoreHook, \Filter and \RecordBuffer.

The \CoreHook is a probe of system call. \CoreHook inspects each system call,   and collects the effects on memory and registers by considering the semantics of system calls. The \Filter stores relevant information of the process what issues the system call, and compares this information with the characteristics specified by the developer. The \RecordBuffer temporarily store the recording of system calls and dumps it to file.

We implement a prototype of \TheName{} and evaluate it with the 
aforementioned requirements in mind. The evaluation results show that
\TheName{} completely records system calls. 
% We also leverage \TheName{} to diagnose 16 failed programs (7 code segments 
% reconstructed from application and 9 real-world applications including Python, Memcached, 
% and SQLite). The diagnosis indicates that \TheName{} effectively identifies the root cause 
% of the failures caused by concurrency and sequential bugs
We also leverage \TheName to record 16 failed programs (7 code segments 
reconstructed from application and 9 real-world applications including Python, Memcached, and SQLite). The recording indicates that \TheName effectively records system calls
 with a performance overhead of
up to 3.88\% on average. Meanwhile, \TheName{} directly works on the unmodified binary of the target 
program and does not rely on any hardware modification.

% The main challenge confornt to \TheName is how to 

% I implement \TheName in three components. The hook component is a couple of callback functions that hooks the entrance and exit of each system call. The filter component are  functions 

In summary, I make the following contributions:
\begin{itemize}
    \item I present a system call recording tool named \TheName{} on Arm platforms, which 
     works with
    unmodified binary on Arm platform without hardware modification.
    %, thus
    %suitable for in-production deployment.
    \item I achieve high performance that allows the
    always-on trace for the production environment, which provides \TheName{} the ability
    to reconstruct the entire records.
    %so \TheName accurately
    %reconstructs control flow based on the trace and the program binary.
    % \item \TheName{} performs \Adaptivedata{} at \Analysisstage{}, and
    % helps developers to reproduce and locate the \bof{}s that occurred in 
    % the production environment.
    %I propose \Adaptivedata to recover complete data flow in
    %\Analysisstage.
    %To achieve this goal, I utilize the records of \Recordingstage
    %to identical re-execute and reproduce bugs. 
    % \TheName adaptively adds
    % hardware watchpoint/breakpoint to produce coredumps to recover data flow.   
    \item I implement a prototype of \TheName{} and evaluate it with
    real-world applications.
    The evaluation result demonstrates that \TheName{} successfully records
    various types of applications with up to 3.88\% runtime
    performance overhead on average.
  \end{itemize}
  