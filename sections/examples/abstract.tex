% !Mode:: "TeX:UTF-8" !TEX program  = xelatex

\begin{英文摘要}{Linux, Syscall, Record}
    Bug diagnosis is difficult. The first step of bug diagnosis is to reproduce the bug. In areas such as application development, developers usually can only rely on the report logs uploaded by the user to try to reproduce bugs.
    Unfortunately, it is still challenging to reproduce bugs that occurred in the production environment at the development environment. The primary obstacle of reproduction is non-deterministic events at runtime, such as system calls. Hence,  the same execution may lead to different results.

    In this thesis, I present \TheName, a practical tool for recording system calls on Linux. \TheName utilizes Linux tracepoints to hook system calls.
    \TheName collects relevant information for each system call related to their effects, which further helps developers reproduce and fix bugs. I implement a prototype of \TheName and evaluate it with real-world applications. The result demonstrates the \TheName capturing system calls completely and efficiently. 
\end{英文摘要}
