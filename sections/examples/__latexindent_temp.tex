\section{Intoduction}
The program would often fail. To sufficiently understand and prevent failures,
developers requires firstly reproduce these bugs, which ensures the same output
and bugs no matter how many times it is re-executed. However, directly
re-exection is not suitable for non-deterministic failures, as they may not
appear in a re-rection procedure. Non-deterministic failures are the consequence
of non-deterministic instructions. 

Instructions for running a program can be divided into two categories. One is
deterministic, i.e., the behavior of the program is determined in each
execution. The other type is non-deterministic, meaning that execution in
different situations will have different results. Although most of the CPU
execution is deterministic, non-deterministic instructions are also pervasive.
This is mainly because of the fact that the execution of a program is not in an
isolated system. In fact, the operating system plays a critical role in program
initialization, system calls, and scheduling throughout the program lifecycle.
Typical sources of nondeterminism include system calls, interrupts, signals, and
data races for concurrency programs.

All these non-deterministic events can be futher classified into two types:
inconstancy of the data flow - for example, certain system calls such as
\texttt{getrandom} and \texttt{getpid}, and inconstancy of the control flow
- for example, concurrency bug due to memory access in inconsistent order. \cite{getrandom2}

Record-and-replay is a type of approaches that addresses this challenge. Most
Record-and-replay systems work by first recording non-deterministic events
during the original run of a program and then substituting these records during
subsequent re-execution. Record-and-replay system could ultimately guarante that
each replay will be identical with the initial version. The fact that a number
of replay systems have been built and put into use in recent years illustrates
the value of record-and-replay systems in practice.\cite{203227}

% There are several ways to capture calls online at runtime: \textbf{PinPlay}, \textbf{REPT}, \textbf{rr}

 .Early record-and-replay systems tend to use virtualization techniques so as to
observe and record the entire program non-deterministically on the hypervisior,
but the virtual machine is very heavy. \cite{dunlap_revirt_2003} Some systems
use dynamic binary instrumentation to get the results after running each
instruction, but this is very inefficient. \cite{bhansali_framework_2006} There
are some other systems that choose not to record at runtime in order to address
the expensive cost of recording; instead, they infer these non-deterministic
events based on the control flow and other information collected.
\cite{altekar_odr_2009,cui_rept_2018}. However, inference often does not
reproduce program execution as faithfully as records, and the time required for
inference, which in the worst case is a search of the entire space, is a
problem. There are also systems that use custom hardware, which inevitably
affects its usefulness in practice. Recently there have been some practical
systems that have adopted tools provided by Linux for tracing, thus achieving
better efficiency. Nevertheless, it still introduces a considerable overhead (50\%) and
is therefore used in scenarios where the developer exactly needs to
debug.\cite{203227}


To solve the challenge of debugging software failures in deployed systems, we argue that we need a practical solution that enables reverse debugging of such failures

In this thesis, I propose \TheName, a practical solution for syscall capturing. It works effectively on Linux systems, on commercial off-the-shelf hardware, and it can run almost imperceptibly along with the original program. There are two key ideas behind \TheName. 

There are two components 

