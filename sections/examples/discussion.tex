\section{Discussion} \label{sec:discussion} 

This section discusses the limitations of \TheName and how I intend to address them in future work.

Currently, \TheName only considers 60 of 276 syscalls. Although I have already accounted for diverse types of system calls, handling more system calls is necessary to make \TheName becoming a practical system. Adding handlers for all system calls is the main task in the future.

Another limitation of \TheName is that it currently only supports system calls. Nevertheless, a similar approach with \TheName can handle other types of non-deterministic events. The
interrupts and traps can be dealt with by similar kernel hooks, and the signals
would require additional capturer in user-space. Moreover, the noninitialized variables are saved in memory, and therefore should appear in the coredump. Therefore, handling interrupts and traps is also one of my future work.

Currently, \TheName works perfectly on ARM devices. In my evaluation, it also runs normally on the x86 platform except for recording the return address. The only difference among these architectures is the approach to obtaining the registers differs. I will add supports for other platforms in the future.

Although \TheName performs well in common scenarios with only 4.88\% overhead at most, I can still apply many optimizations. For example, different record formats for different system calls are still a problem, and my current solution is to name each item. Therefore, records likes \texttt{pid=14311, accept4, res=3} (with the attribute name \texttt{pid} and \texttt{res}) additionally consumes a considerable amount of space. In the future, I need to classify the format of the record to reduce the space occupation of records.


In addition, \Filter of \TheName can only support limited conditions. To make \TheName more practical, some basic logic operators (e.g., \texttt{AND}, \texttt{OR} and \texttt{NOT}) should also be supported.

Finally, \TheName needs to have the ability to support long-term running. The size of the record file should not increase unlimitedly. Therefore, processing of historical recording data (dumping, compression, overwriting, etc.) will also be a future work.