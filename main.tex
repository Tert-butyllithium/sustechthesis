% !Mode:: "TeX:UTF-8"
% !TEX program  = xelatex
\documentclass[AutoFakeBold,AutoFakeSlant,scheme=plain,degree=bachelor,zihao=-4]{sustechthesis}
% 1. AutoFakeBold 与 AutoFakeSlant 为伪粗与伪斜,如果本机上有相应粗体与斜体字体,请使用 xeCJK 宏包进行设置,例如:
%   \setCJKmainfont[
%     UprightFont = * Light,
%     BoldFont = * Bold,
%     ItalicFont = Kaiti SC,
%     BoldItalicFont = Kaiti SC Bold,
%   ]{Songti SC}
%
% 2. scheme=chinese 为 ctexart 文类提供的中文排版方案,如果使用英文进行论文创作,请使用 scheme=plain 选项。
%
% 3. degree=bachelor 为 sustechthesis 文类提供的本科生毕业论文模板,其他可选项为 master 与 doctor,但是均未实现,如果您对此有兴趣,欢迎 PR。
%
% 4. sustechthesis.cls 文类主要参考自去年完成使命的 sustechthesis.tex,在这一年的时间,作者的 TeX 风格与常用宏包发生许多变化,因为之前的思想为仅提供必要的格式修改相关代码,所以转换为文类形式所进行的修改较少,而近期的风格与常用宏包均体现在以下的例子文件中。
%
% 5. 示例文件均放置于相应目录的 examples 文件夹下,构建自己论文时可暂时保留,用以检索接口与使用方法。
%
% 6. 英文目录需要居中可以使用:\renewcommand{\contentsname}{\centerline{Content}}
%
% 7. LaTeX 中公式编号括号样式及章节关联的方法:https://liam.page/2013/08/23/LaTeX-Formula-Number/

% !Mode:: "TeX:UTF-8"
% !TEX program  = xelatex

% 数学符号与环境
\usepackage{amsmath,amssymb}
  \newcommand{\dd}{\mathrm{d}}
  \newcommand{\RR}{\mathbb{R}}
% 参考文献
\usepackage[style=gb7714-2015]{biblatex}
  \addbibresource{ref.bib}
% 无意义文本
\usepackage{zhlipsum,lipsum}
% 列表环境设置
\usepackage{enumitem}
% 浮动题不越过 \section
\usepackage[section]{placeins}
% 超链接
\usepackage{hyperref}
% 图片,子图,浮动题设置
\usepackage{graphicx,subcaption,float}
% 抄录环境设置,更多有趣例子请命令行输入 `texdoc tcolorbox`
\usepackage{tcolorbox}
  \tcbuselibrary{xparse}
  \DeclareTotalTCBox{\verbbox}{ O{green} v !O{} }%
    {fontupper=\ttfamily,nobeforeafter,tcbox raise base,%
    arc=0pt,outer arc=0pt,top=0pt,bottom=0pt,left=0mm,%
    right=0mm,leftrule=0pt,rightrule=0pt,toprule=0.3mm,%
    bottomrule=0.3mm,boxsep=0.5mm,bottomrule=0.3mm,boxsep=0.5mm,%
    colback=#1!10!white,colframe=#1!50!black,#3}{#2}%
\tcbuselibrary{listings,breakable}
  \newtcbinputlisting{\Python}[2]{
    listing options={language=Python,numbers=left,numberstyle=\tiny,
      breaklines,commentstyle=\color{white!50!black}\textit},
    title=\texttt{#1},listing only,breakable,
    left=6mm,right=6mm,top=2mm,bottom=2mm,listing file={#2}}

% LaTeX logo
\usepackage{hologo}
% code highlight
% \usepackage{minted}
\usepackage{abbrevs}
\makeatletter
\renewcommand\maybe@space@{%
  \maybe@ictrue % <= this is new
  \expandafter   \@tfor
    \expandafter \reserved@a
    \expandafter :%
    \expandafter =%
                 \nospacelist
                 \do \t@st@ic
  \ifmaybe@ic % <= this is new
    \space
  \fi
} % 导言区
% !Mode:: "TeX:UTF-8"
% !TEX program  = xelatex
\设置信息{
%   键 = {{中文值}, {英文值}},
    分类号 = {{}, {}},
    编号 = {{}, {}},
    UDC = {{}, {}},
    密级 = {{}, {}},
    题目 = {{}, {\scalebox{0.84}{Real-time Capturing of System Calls on ARM}}},
    子标题 = {{}, {}},
    姓名 = {{}, {Haonan Li}},
    学号 = {{}, {11712510}},
    系别 = {{}, {Computer Science and Engineering}},
    专业 = {{}, {Computer Science and Technology}},
    指导老师 = {{}, {Fengwei Zhang, Associate Professor}},
    时间 = {{}, {2021.6.3}},
}
 % 论文信息


\begin{document}

\英文标题页
\英文诚信承诺书
\摘要标题
% !Mode:: "TeX:UTF-8" !TEX program  = xelatex

\begin{英文摘要}{Linux, Syscall, Record}
    Bug diagnosis is difficult. The first step of bug diagnosis is to reproduce the bug. In areas such as application development, developers usually can only rely on the report logs uploaded by the user to try to reproduce bugs.
    Unfortunately, it is still challenging to reproduce bugs that occurred in the production environment at the development environment. The primary obstacle of reproduction is non-deterministic events at runtime, such as system calls. Hence,  the same execution may lead to different results.

    In this thesis, I present \TheName, a practical tool for recording system calls on Linux. \TheName utilizes Linux tracepoints to hook system calls.
    \TheName collects relevant information for each system call related to their effects, which further helps developers reproduce and fix bugs. I implement a prototype of \TheName and evaluate it with real-world applications. The result demonstrates \TheName capturing system calls completely and efficiently. 
\end{英文摘要}
 % 论文摘要

\目录\clearpage % 目录及换页

\section{Intoduction}
The program often fails. To sufficiently understand and prevent failures,
developers requires firstly reproduce these bugs, which ensures the same output
and bugs. However, directly
re-exection is not suitable for non-deterministic failures, as they may not
appear in a re-execution procedure. Non-deterministic failures are the
consequence of non-deterministic instructions. 

Instructions for running a program can be divided into two categories. One is
deterministic, which means the behavior of deterministic instruction is determined in each
execution. The other type is non-deterministic, meaning that execution in
different situations will have different results. Although most of the CPU
execution is deterministic (e.g., \texttt{ADD}), non-deterministic instructions (e.g., get user input) are also pervasive.
Typical sources of nondeterminism include system calls, interrupts, signals, and
data races for concurrency programs \cite{ronsse_recplay_1999}. All these non-deterministic events can be futher classified into two types:
inconstancy of the data flow - for example, certain system calls such as
\texttt{getrandom} and \texttt{getpid}, and inconstancy of the control flow
- for example, concurrency bug due to memory access in inconsistent order \cite{getrandom2}.

Record-and-replay is a type of approaches that addresses this challenge. Most
Record-and-replay systems work by first recording non-deterministic events
during the original run of a program and then substituting these records during
subsequent re-execution. Record-and-replay system could ultimately guarante that
each replay will be identical with the initial version. The fact that a number
of replay systems have been built and put into use in recent years illustrates
the value of record-and-replay systems in practice \cite{203227,replay_survey,altekar_odr_2009,bhansali_framework_2006}.

% There are several ways to capture calls online at runtime: \textbf{PinPlay}, \textbf{REPT}, \textbf{rr}

There is a rich amount of research on record-and-replay systems, and we can find
their various treatments of non-deterministic records. Early record-and-replay
systems tend to use virtualization techniques so as to observe and record the
entire program non-deterministically on the hypervisior, but the virtual machine
is very heavy \cite{dunlap_revirt_2003, dunlap_smp-revirt_2008}. Some systems
use dynamic binary instrumentation to get the results after running each
instruction, but this is very inefficient \cite{bhansali_framework_2006}. There
are some other systems that choose not to record at runtime in order to address
the expensive cost of recording; instead, they infer these non-deterministic
events based on the control flow and other information collected
\cite{altekar_odr_2009,cui_rept_2018}. However, inference often does not
reproduce program execution as faithfully as records, and the time required for
inference, which in the worst case is a search of the entire space, is a problem
\cite{replay_survey}. There are also systems that use custom hardware, which
inevitably affects its usefulness in practice \cite{montesinos_capo_2009}.
Recently there have been some practical systems that have adopted tools provided
by Linux for tracing, thus achieving better efficiency. Nevertheless, it still
introduces a considerable overhead (50\%) and is therefore used in scenarios
where the developer exactly needs to debug \cite{203227}.

This thesis focuses on the data record part of record-and-replay systems, precisely, the recording of non-deterministic events caused by system calls. I argue that a \textit{practical} record system should  (1) run online, meaning that the recording has little performance impact on the execution of the target program, (2) log all data without any omission, (3) work on commercial off-the-shelf hardware, (4) not require any modification to the target program, and also (5) not require any modification to the kernel.

In this thesis, I propose \TheName, a practical solution for syscall capturing. 
It works with unmodified Linux programs on commercial off-the-shelf (OTS) hardware. My original design was on the ARM platform, but the system can be applied to other platforms as well (e.g. x86, riscv). I demonstrat its usefulness on both x86 and ARM platforms. \TheName consists of three component: \CoreHook, \Filter and \RecordBuffer.

The \CoreHook is a probe of system call. \CoreHook inspects each system call,   and collects the effects on memory and registers by considering the semantics of system calls. The \Filter stores relevant information of the process what issues the system call, and compares this information with the characteristics specified by the developer. The \RecordBuffer temporarily store the recording of system calls and dumps it to file.

We implement a prototype of \TheName{} and evaluate it with the 
aforementioned requirements in mind. The evaluation results show that
\TheName{} completely records system calls. 
% We also leverage \TheName{} to diagnose 16 failed programs (7 code segments 
% reconstructed from application and 9 real-world applications including Python, Memcached, 
% and SQLite). The diagnosis indicates that \TheName{} effectively identifies the root cause 
% of the failures caused by concurrency and sequential bugs
We also leverage \TheName to record 16 failed programs (7 code segments 
reconstructed from application and 9 real-world applications including Python, Memcached, and SQLite). The recording indicates that \TheName effectively records system calls
 with a performance overhead of
up to 3.88\% on average. Meanwhile, \TheName{} directly works on the unmodified binary of the target 
program and does not rely on any hardware modification.

% The main challenge confornt to \TheName is how to 

% I implement \TheName in three components. The hook component is a couple of callback functions that hooks the entrance and exit of each system call. The filter component are  functions 

In summary, I make the following contributions:
\begin{itemize}
    \item I present a system call recording tool named \TheName{} on Arm platforms. To the
    best of our knowledge, it is the first hardware-assisted system call recording tool
    for Arm architecture. 
    \TheName{} works with
    unmodified binary on Arm platform without hardware modification.
    %, thus
    %suitable for in-production deployment.
    \item I achieve high performance that allows the
    always-on trace for the production environment, which provides \TheName{} the ability
    to reconstruct the entire records.
    %so \TheName accurately
    %reconstructs control flow based on the trace and the program binary.
    % \item \TheName{} performs \Adaptivedata{} at \Analysisstage{}, and
    % helps developers to reproduce and locate the \bof{}s that occurred in 
    % the production environment.
    %I propose \Adaptivedata to recover complete data flow in
    %\Analysisstage.
    %To achieve this goal, I utilize the records of \Recordingstage
    %to identical re-execute and reproduce bugs. 
    % \TheName adaptively adds
    % hardware watchpoint/breakpoint to produce coredumps to recover data flow.   
    \item I implement a prototype of \TheName{} and evaluate it with
    buggy real-world applications.
    The evaluation result demonstrates that \TheName{} successfully records
    various types of applications with up to 3.88\% runtime
    performance overhead on average.
  \end{itemize}
  
\section{Background}

\subsection{Linux Tracepoint}



 Tracepoint is a lightweight hook for events provided by Linux. With tracepoint, we can provide a callback function (called a probe) at runtime for Linux to execute when these events are triggered. \cite{torvalds_torvaldslinux_2021,using_tracepoint}

There are two events (\texttt{sys\_enter} and \texttt{sys\_exit}) defined in \texttt{/include/trace/} \texttt{events/syscalls.h} that can be exploited to execute callback function in the procedure of invloking syscall. \cite{torvalds_torvaldslinux_2021}

\subsection{Linux Process Management}

Linux uses process descriptors to represent all information about each process, such as the process priority, the address space allocated, the files allowed to be accessed, and so on. Process descriptors are all of type \texttt{task\_strcut}.  \cite{torvalds_torvaldslinux_2021,bovet_cesati_2006}

In Linux kernel, there is a macro named \texttt{current} defined in \texttt{arch/arm64/include/} \texttt{asm/current.h} (on Arm64). \texttt{current} is a pointer that refers to the process that is currently executing; precisely, the issuer of the system call. \cite{corbet_linux_2005,torvalds_torvaldslinux_2021}
\section{Related Work}

There is a large amount of related work to cpaturing syscalls. In this section, I discuss some representative examples and describe
how \TheName differs.

\textbf{Record and replay systems} are a type of systems that record sufficient information to reconstruct the program execution and then replay this execution. Record and replay systems are undoubtedly necessary to handle system calls or not replay faithfully.
Pinplay \cite{patil_pinplay_2010} is a record and replay system based on Pin  \cite{reddi_pin_2004}, a binary instrumentation system. While it does record system calls, the instrument-based approach introduces significant overhead (up to 140x). REPT \cite{cui_rept_2018} chooses to not record non-deterministic events such as system calls. In contrast, REPT adopts inference algorithms to reconstruct the effects of system calls. However, inference-based approaches can not achieve a 100\% reconstruction. For example, REPT can not deal with bugs like Figure \ref{fig:data-race}.

RR \cite{203227} is one of the-state-of-the-practice record and replay systems. RR perform the recording to non-deterministic events via \texttt{ptrace} \cite{ptrace2},
a \syscall{}
that allows a process to inspect and control the execution of another
one.
By using \texttt{ptrace}, a supervisory process can easily observe and intercept \syscall{}s issued by the target program.
Nevertheless,
% \zhenyu{what is "the supervisor"?}
\texttt{ptrace} brings a considerable overhead (up to 7.85×) as the supervisor also runs in
user space, 
% \zhenyu{why you write this? You introduced a tool, and then suddenly
% talk about some other things.}
which means that the supervisor needs to capture data with several
context switches (between target process with supervisor).

% \textbf{Linux troubleshooting tools} are a type of tools to inspect and record system status. Some Linux troubleshooting tools 
Some \textbf{Linux troubleshooting tools} monitor the target application and captures data
from non-deterministic events, such as Sysdig \cite{github_sysdig_2021}, DTrace
\cite{gregg_dtrace_2019,gregg_dtrace_2011}, and Strace \cite{github_strace_2021}.
Nevertheless, Strace \cite{github_strace_2021} also utilize \texttt{ptrace} to capture syscalls, which is similar to RR and inevitably introduces a considerable overhead. DTrace is powerful and efficient, but it is too sophisticated to use and require technical knowledge to optimize \cite{gregg_dtrace_2011}. Sysdig also leverages \textit{Linux tracepoints} as \TheName to achieve a relevant reasonable overhead (about 15\%); however, Sysdig sacrifices the completeness of its records, i.e., it does not guarantee that all system calls will always be recorded \cite{degioanni_sysdig_2014}.
% Nevertheless, these tools are desgined to
% monitor and troubleshoot applications, and performance is not their primary
% consideration.
% Nevertheless, some of these tools are designed to monitor and troubleshoot applications
% instead of precisely recording their effects \cite{github_sysdig_2021}; some do not concern about the runtime performance ; 
% others are hard to use and require technical knowledge to optimize \cite{gregg_dtrace_2011}.


% \begin{itemize}
%     \item \textbf{Pinplay} \dots
%     \item \textbf{REPT} \dots
%     \item \textbf{rr} \dots
%     \item \textbf{DTrace} \dots
%     \item \textbf{sysdig} \dots
% \end{itemize}


\section{Conclusion}

I propose a syscall record tool \TheName on ARM to satisfy practical requirements. It takes advantage of the mechanisms in Linux to completely and efficiently capture and record system calls. \TheName achieves its design goals and is suitable as the basis for record and replay systems. 
\section{Design}\clearpage
\参考文献
\nocite{*}
  \printbibliography[heading=none]
  \clearpage
% \附录
%   % !Mode:: "TeX:UTF-8"
% !TEX program  = xelatex

% \section*{数据获取函数}\label{A:data}
% \Python{utils.py}{code/examples/utils.py}
\section*{Source Code}
\TheName is an open source proejct on Github: its source code can be found at https://github.com/Tert-butyllithium/syscord\clearpage
% \致谢
  % \TheName is an important part of another project, which is submitted to ACM CCS 2021. Thanks to my collaborators, including Yiming Zhang, Wenxuan Shi, Yuxin Hu, and Xueying Zhang. I have learned a lot from the project. Besides, Yiming as the first user of \TheName has also raised many issues for \TheName, which makes \TheName better.

I would like to express my gratefulness to my advisor, Dr. Fengwei Zhang, for his continuous support and encouragement during my college life in SUSTech.  Dr. Zhang's encouragement has rekindled my passion for research and eventually motivated me to continue on my research path. Dr. Zhang provided a lot of time to discuss with me when I was facing difficulties in research and life.

I am very appreciative to Dr. Zhenyu Ning for his guidance in our project. Dr. Ning's insightful and critical suggestions are an essential factor in making this project successful. Dr. Ning's enthusiasm, modest and gentle, has also influenced me in my life and research.

Moreover, I would like to thank all the group members in COMPASS Lab. I have joined COMPASS Lab a year ago and it has been a pleasant and impressive experience with every genius in the lab. 

Meanwhile, I treasure my nearly two years of training at SUSTech-CPC, and I would like to thank every member of the training team and my coach, Dr. Bo Tang. Although I didn't achieve the expected results, the continuous practice has improved my comprehensive programming proficiency. More importantly, I have encountered many good friends in the team, who gave me a lot of support and encouragement and helped me get through the most miserable period.



Also, I deeply appreciate to my girl friend, who has been absent for 21 years of my life. Therefore I can only focus on my work. 
\end{document}
